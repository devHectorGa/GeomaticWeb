%Para hacer informe con portada utilizamos report

\documentclass[12pt]{report}

\usepackage[a4paper]{geometry}
\usepackage[myheadings]{fullpage}
\usepackage{fancyhdr}
\usepackage{lastpage}
\usepackage{graphicx, wrapfig, subcaption, setspace, booktabs}
\usepackage[T1]{fontenc}
\usepackage[font=small, labelfont=bf]{caption}
\usepackage{fourier}
\usepackage[protrusion=true, expansion=true]{microtype}

%Paquete para hipervinculos
\usepackage[colorlinks=true]{hyperref}
\hypersetup{
    colorlinks=true,
    linkcolor=black,
    filecolor=magenta,      
    urlcolor=blue,
}

%Para qué los subtítulos aparezcan en español
\usepackage[spanish]{babel}
\usepackage[utf8]{inputenc}
\usepackage{sectsty}
\usepackage{url, lipsum}
\usepackage{tabularx}
\usepackage{float}

%--------------------------------------------------
%Para agregar citas en apa
%Para citar se usa el comando \cite{}
%Las referencias se modifican en el archivo sample.bib
\usepackage{apacite}
%----------------------------------------------

\newcommand{\HRule}[1]{\rule{\linewidth}{#1}}
\onehalfspacing
\setcounter{tocdepth}{5}
\setcounter{secnumdepth}{5}

%-------------------------------------------------------------------------------
%Encabezado y pie de pagina y numeracion
%\fancyhead para encabezado
%\fancyfoot para pie de pagina
% L para izquierda, left
% R para derecha, right
% C para centro, center
%-------------------------------------------------------------------------------
\pagestyle{fancy}
\fancyhf{}
\setlength\headheight{15pt}
\fancyhead[L]{\chaptername \ \thechapter} 
\fancyhead[R]{Universidad de Costa Rica}
\fancyfoot[R]{\thepage}

\begin{document}


%-------------------------------------------------------------------------------
% Portada
%-------------------------------------------------------------------------------
\title{ \normalsize Universidad de Costa Rica \\
		Facultad de Ecucación\\
		Programa de Tecnologías en Educación Avanzada
		\\ [2.0cm]
		\HRule{0.5pt} \\
		\LARGE \textbf{Propuestas de Taller de \LaTeX} %para que quede encerrado en las lineas
		\HRule{2pt} \\ [0.5cm]
		\normalsize \today \vspace*{5\baselineskip}}

\date{}

\author{
		Sofía Fonseca Muñoz \\ 
		 sofia.fonsecamunoz@ucr.ac.cr}
%se debe incluir el comando \maketitle para hacer 
\maketitle

%Para hacer el indice solo es necesario agregar 
\tableofcontents
\newpage

%-------------------------------------------------------------------------------
% Section title formatting
%\sectionfont{\scshape}
%-------------------------------------------------------------------------------

%-------------------------------------------------------------------------------
% BODY
%-------------------------------------------------------------------------------

\chapter{Introducción}

\section{Justificación}
Necesitamos cambiar el nombre de la letra, en \url{https://es.wikibooks.org/wiki/Manual_de_LaTeX/Escribiendo_texto/Tamaños,_estilos_y_tipos_de_letra}

\section{Objetivos}
\subsection{Objetivo general}
Utilizar la herramienta \LaTeX para el desarrollo de instrumentos de investigación en el área de la educación.
\subsection{Objetivos especificos}
\begin{itemize}
\item Identificar Latex como una herramienta útil para la confección del formato de una propuesta de investigación.
\item Reconocer las plantillas de \LaTeX como ayuda para documentar.
\item Conocer los comandos básicos de \LaTeX para la adaptación de plantillas en trabajos propios.
\end{itemize}

\section{Competencias}
Al finalizar este taller, se espera que el participante sea capaz de:
\begin{itemize}
\item Entender la utilidad de la herramienta Latex el desarrollo del formato de propuestas de investigación.
\item Adaptar plantillas de Latex para la creación de su propia documentación.
\item Insertar sus propias tablas, imágenes y secciones en documentos de \LaTeX.
\end{itemize}

\section{Contenido} 
El contenido de este taller se encuentra en las presentaciones adjuntas que se complementan con el material de apoyo presentes en el capítulo 2 de este documento.\cite{greenwade93}

Además vemos como Einstein habla \cite{Einstein}
\section{Cronograma}
\subsection{Día 1}
%--------------------------------TABLAS-------------
%Insertar tabla
% Las tablas se pueden genederar a través de:
% www.tablegenerator.com
%---------------------------------------------------

\newpage
\chapter{Material de apoyo}
\section{Links útiles}
\begin{itemize}
\item Instalador para MacOS:
http://www.tug.org/mactex/mactex-download.html
\item Compiladores en línea:
\item Generador de tablas:
\item Manuales de ayuda:
\end{itemize}


%-------------------------------------------------------------------------------
% REFERENCIAS
%-------------------------------------------------------------------------------
\newpage

\bibliographystyle{apacite}
\bibliography{sample.bib}


\end{document}

%-------------------------------------------------------------------------------
% SNIPPETS
%-------------------------------------------------------------------------------

%\begin{figure}[!ht]
%	\centering
%	\includegraphics[width=0.8\textwidth]{file_name}
%	\caption{}
%	\centering
%	\label{label:file_name}
%\end{figure}

%\begin{figure}[!ht]
%	\centering
%	\includegraphics[width=0.8\textwidth]{graph}
%	\caption{Blood pressure ranges and associated level of hypertension (American Heart Association, 2013).}
%	\centering
%	\label{label:graph}
%\end{figure}

%\begin{wrapfigure}{r}{0.30\textwidth}
%	\vspace{-40pt}
%	\begin{center}
%		\includegraphics[width=0.29\textwidth]{file_name}
%	\end{center}
%	\vspace{-20pt}
%	\caption{}
%	\label{label:file_name}
%\end{wrapfigure}

%\begin{wrapfigure}{r}{0.45\textwidth}
%	\begin{center}
%		\includegraphics[width=0.29\textwidth]{manometer}
%	\end{center}
%	\caption{Aneroid sphygmomanometer with stethoscope (Medicalexpo, 2012).}
%	\label{label:manometer}
%\end{wrapfigure}

